\section{Animals}
\label{sec:Animals}

All procedures were approved by the Champalimaud Centre for the Unknown Ethics Committee and carried under the stipulations of the Portuguese Direção Geral de Veterinária. Mice were held in individual cages on a reversed light-dark cycle with access to food and water. Mice were exclusively used for the experiments regarding this thesis' work.

Cells' somata in V1 layer 2/3 of four Thy1-GCaMP6s male mice were imaged.

Prior to the imaging experiments, once adults, the mice underwent chronic window implantation surgeries. A circular craniotomy of diameter $4mm$ was performed over each mouse's left visual cortex, leaving the dura intact. The imaging windows were constructed using two layers of microscope cover glass (Fisher Scientific, no. 1 and no. 2) and UV-curable optical glue. A window was placed into the craniotomy using black dental cement and an iron headpost was attached to the skull with dental acrylic. The subjects were kept under isoflurane anesthesia, as well as Bupivacaine ($0.05\%$; injected under the scalp) and Dolorex ($1 mg/kg$; injected subcutaneously), serving respectively as local and general analgesia. Eye moisturing was insured with ophtalmic ointment (Clorocil, Laboratorio Edol).

For both the ISOI and visual stimuli protocols, the animals were lightly anesthetized with isoflurane ($1\%$) and injected intramuscularly with chlorprothixene (1mg/kg), a muscular paralyzer to circumvert the need for higher anesthesia concentrations which could depress the recorded neuronal responses. Mice were headfixed by the headposts during all of the visual stimuli presentations and their eyes were protected and kept moist with silicone oil (Sigma-Aldrich) in thin, uniformly coated layers.
