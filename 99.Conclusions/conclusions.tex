% %%%%%%%%%%%%%%%%%%%%%%%%%%%%%%%%%%%%%%%%%%%%%%%%%%%%%%%%%%%%%%%%%%%%%%
% The Introduction:
% %%%%%%%%%%%%%%%%%%%%%%%%%%%%%%%%%%%%%%%%%%%%%%%%%%%%%%%%%%%%%%%%%%%%%%
\fancychapter{Conclusions and Future Work}
\label{cap:conclusions}


\section{Surround modulation remaining questions and project's plan} \label{Plan}

In a cell, SM might not be isotropic and furthermore, its anisotropy may depend on the stimulus' nature, in general or relative to the regarded neuron's selectivity. However, most previous experiments on SM have been devised with circular stimuli varying in radius, assuming the effect's isotropy. The present project proposes to tackle the possible anisotropy itself.

Furthermore, so far, studies have mostly regarded individual neuron's SM properties in cat or monkey subjects. In mice, neurons with the same orientation selectivity do not lie in the same column. Neurons with different tunings are mixed. Since this thesis task will be conducted in mice, we can thus regard multiple neurons with different tunings at once, in a more complete approach.

A recent procedure has also been done with differently localized patches of gratings in the surround \cite{SManisotropy}. However, this was conducted with static stimuli, and was furthermore executed with limited stimuli configurations. Besides using moving gratings, this project's aim is to produce an extensive study on a multitude of neurons, tunings, and stimuli configurations: Only in this way can we pursue the finding of a set of more generalized and integral SM spatial structure rules.

Having these objectives in mind, experimental and analytical steps will comprise, for each mouse: 
\begin{enumerate}
    \item Imaging large populations of neurons (>1000).
    \item Measuring the neurons' receptive fields. These will be different from cell to cell. 
    \item Suiting the larger possible number of neurons, appropriate stimulus RF and surround sizes will be fixed. Only the neurons whose receptive field does not intersect the stimuli surround will be analyzed.
    \item A central stimulus will be presented. This stimulus will be a moving grating that can have 4 different cardinal directions, thus exciting different neurons, according to their selectivities. First recordings will be for only this center stimuli (4 possibilities).
    
    \item For each of these RF's stimuli displays, small patches of moving gratings will be simultaneously presented in the surround, under a variety of configurations:
    
        \subitem{Only one patch, that can be presented in 4 angular positions of the ring-shaped surround (N, E, S, W). For each of these locations, surround gratings' will have the two directions and 2 orthogonal orientations (64 possibilities).}
        
        \subitem{Two patches located around the surround. These can be located in a vertical or horizontal direction line. For each of these configurations, the patches grating can further be with the same orientation or with opposite directions - no other relative orientation will be presented (16 possibilities).}
    
    \item Control recordings will also be operated only with surround gratings, to ensure that these are not directly driving the neuron and are by definition displayed in the surround (16 possibilities).
        
    \item The large population of neurons' responses will then be analyzed, for each of these 100 possible configurations. Comparison across neurons and across configurations will then take place, as to produce a set of SM spatial structure rules and discover possible diversity and asymmetries in neurons' responses to stimuli. 
    

\end{enumerate}

Conclusions Chapter

\cleardoublepage