\section{Thesis Outline}
\label{sec:int_outline}

In the following section \ref{cap:TheoreticalIntroduction}, visual neuroscience main state of the art notions are described, to introduce the relevant theoretical concepts. 

In the appendix chapter \ref{cap:Technology}, we review the chemical indicators, genetic manipulation and physical mechanisms used for this experiment's expression of neural activity, as well as the recording optical technologies, namely intrinsic signal optical imaging and two-photon excitation laser-scanning microscopy.

Chapter \ref{cap:TechnicalImplementations} describes and explains the software development phase and implemented system for stimuli display, elucidating the technical outcomes of the project.

Chapter \ref{cap:ExperimentalMethods} details the experimental methods, starting with the procedures performed with the animals, following with the visual stimulation setups for both intrinsic imaging and two-photon microscopy, then describing the experimental sessions' pipelines, stimuli presentation structure and stimuli parameters for the three protocols (RF, tuning and SM spatial structure study).

The analysis of the experimental data required appropriate methods for the treatment of great amounts of possible parameter configurations, each for a large number of individual cells. In chapter \ref{cap:Analysis}, first we enunciate the raw outputs of an experimental session and follow with the preliminary image processing stages applied to the recorded imaging data - plane separation and registration. Suit2p pipeline is then presented, as it was used for semi-automated selection of regions of interest (ROIs) to classify as neurons. Normalized fluorescence extraction is explained. Then, RF mapping methods and some example results are shown, along with the tuning analysis description and examples. Finally, some precluding considerations are made on the analysis methods for the final SM results.

A review is also presented, on appendix chapter \ref{chapter:StatisticalTests}, on the statistical notions applied for the SM data comparison analysis.

This project's results are then described in chapter \ref{cap:Results}. We start with SM results analysis of individual example cells and finish with a broader population study for SM effects.

Finally, a discussion on the findings follows in chapter \ref{cap:conclusions}, integrating the previously established SM properties across species and sensory areas with those hereby presented and contextualizing this project's results within the SM current theories. We conclude with an overall review of the project's process.

%In section \ref{Plan}, the experimental previous methods on the matter of SM are also regarded, motivating this proposal's goals and novelty. 

%The detailed experimental procedures this project aims will then be established as to answer remaining questions about SM. This document then accounts for the relevant experimental and analytical technologies, in section \ref{Experimental}: First, an explanation is provided on behaviour measurement control settings and a programmed stimuli display protocol, as well as the further adaptations it shall undergo for this project. Secondly, the relevant principles in chemical indicators and optical physics mechanisms to use for the expression and detection of neural activity will be clarified. Finally, the analysis of the experimental data will require appropriate methods for the treatment of a large number of individual cells - We will present a possible working approach to this, Suit2p. This report will then follow with a commented bibliography, section \ref{Bib}, the calendarization of the thesis work, section \ref{Cal}, and a final discussion of the general project's plan overview, section \ref{Con}.

%\\In the following state of the art section, as a theoretical introduction, we will start by presenting visual neuroscience main general notions (section 2.1), and then specify the current understanding of the retina's neurophysiology (section 2.2) and visual cortex (section 2.3). We will then enunciate the perception organization central ideas (section 2.4) and finally review the modern surround modulation investigation results, particularizing for V1 (section 2.5).

%%other principles in neurobiology are relevant, as well as computation familiarity with Matlab, $C^{++}$, specific behavior control and measurement packages and extensions, as well as optical physics and instrumentation engineering fundaments. These will be delineated
%\\The experimental phases of this project, will be delineated starting with a section for the stimuli and software development programming (section 3.1), a review on the sort of chemical indicators, genetic manipulation and physical mechanisms used for this experiment's expression of neural activity (section 3.2) and finalized by a presentation of the recording optical technology, two-photon excitation microscopy (section 3.3).

%\\Finally, the analysis of the experimental data will require appropriate methods for the treatment of great amounts of possible parameter configurations, each for a large number of individual cells - We will present a possible working approach to this, Suit2p (section 4).

%This report will follow with a commented bibliography (section 5), the calendarization of the to-be-done work (section 6) and a final discussion of the general project's plan overview (section 7).