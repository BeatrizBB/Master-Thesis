\section{A systems neuroscience question: How does the nervous system perceive the visual world?}
\label{sec:int_motivation}

Neuroscience strives to understand the organization of the nervous system, it's function and processes, as well as its relation to animal behaviour. These broad goals require a multidisciplinary integration of concepts and tools derived from biology, biophysics, anatomy, genetics, statistics, modelling, computation, ecology and psychology. Furthermore,these questions can be approached at multiple scales: from the molecular and cellular levels to the systems and neural circuits levels.

Neurons and glia, the fundamental cells in a nervous system, agglomerate in neural circuits. In the human brain, there are about $10^11$ neurons, of a great morphological and functional variety. Moreover, neurons are adaptive cells: each neurons can behave differently depending on the connections and signals it receives and transmits. 
Thus, disentangling the fundamental parameters of neuronal complexity and understanding the function of the nervous system require not only the signals and individual cell biophysical comprehension but also the understanding of the connectivity circuitry and mechanisms underlying a larger-scale neuronal network. Two systems with the same cells, arranged and interleaved with different connections would not behave equally. Correspondingly, the established knowledge on molecular and cellular neuroscience shall be integrated and complemented by a systems neuroscience approach, not as a simpler scaling step but within fundamentally different strategies. Investigating and deciphering the extraordinary questions about the nervous system in higher order completeness demand different methods, analysis and experimental paradigms than those within the study of the sum of its parts.

Neuronal systems can be described in three main functional sets: Sensory systems, motor systems and associational systems that link the former with the latter, developing more complex cognitive processes, such as perception, attention and emotions.

Animals have the enriching ability to sense the world around them. We sense our surroundings through efficiently designed stimuli sensors, and produce distinctive sensations accordingly. We are equipped with exceptionally sensible, accurate and complete input feature detectors.

However, perhaps one of the most fascinating qualities about a sensorial experience is that we can mentally encapsulate it as a given configuration and readily identify it at a future similar reoccurrence. Our nervous system allows the formation of a correspondence map between the world outside and the interior reality. The computational processing level and the efficiency of such endeavour continues to excite laborious research: How does perception arise?

In particular, the process of \textit{seeing} pertains astoundingly substantial and relevant information about the world in a remarkably efficient, fast, detailed and engaging way.

The understanding of visual perception and its processes is undertaken as one of the major withstanding challenges in systems neuroscience.

In this case, the system receives the physical image information from the retina and then parallel processes the features from the current state of the visual scene in an hierarchically organized neuronal structure. The information signals follow feedforward pathways and integrations are then carried in higher brain areas. Concurrently, feedback connections are superimposed in multiple higher order to lower order area connections, transmitting signals that underlie contextual information. Receiving this higher complexity input, neurons can then change their functioning state and produce new conjectures, educated guesses of high success probability about the input's nature. 
The unification of these parallel outputs is proposed to finally amount to a conscious sensorial perception.

There is no identity complete copy of the world outside within our perceived reality - Nonetheless, perhaps contrary to our intuition, sensitive sophisticated guesses can be just as effective, while efficient and biologically feasible within nature's parsimonious frames.

Nonetheless, fundamental questions remain for the large and complete understanding of sensory perception. In particular, discovering the computations held in the sensory cortices, their functionality, as well as the circuitry substrates that serve these mechanisms stand as main goals.

Here, we tackle a sensory neuronal property that is fundamentally implicated in visual information processing and that can yield profound insight onto the neuronal circuitry leading to visual perception - surround modulation (SM).

Neurons in the primary visual cortex (V1) respond to visual stimuli when it is presented in a given region of the visual field. For each visually responsive neuron, there is thus a \textit{receptive field} (RF) - presenting a visual stimulus of optimal parameters to that neuron in the RF corresponding visual region, will evoke a spiking response. By definition, presenting stimuli outside the RF of a neuron will amount to no response in that cell.
However, the simultaneous presentation of stimuli inside and outside the RF of a neuron will lead to a modulation of the cell's signal: the response will be either suppressed or facilitated in regards to the RF-only condition, depending on the center and surround stimuli characteristics.

This property is always active during vision processes, has been described in several species (mouse, cat, primates, humans), in various visual system areas (from the retina to the extrastriate cortex) and in multiple sensory modalities (visual, auditory, somatosensory, olfactory). A mechanistic SM theory would in this way provide a manifestly important framework in which to understand sensory processing. Moreover, understanding the circuitry that results in the SM effect and accounts for its properties can lead to insight about the function and the organization of the involved connections and network patterns - in particular, SM characterization can aid restrain circuitry models and interpret the network of inhibitory and excitatory neurons as well as feedforward, feedback and horizontal connections functionality in that perceptual integration context.

Most specifically, \textit{far} surround modulation has been suggested to require feedback input.

Typically, SM is studied in either of two methods: By using radius expanding circular grating patches or by using grating patches confined to the RF area and surrounded with annular gratings whose inner radius are made to decrease. These studies have lead to important findings about the properties of SM and elicited fruitful debate on the circuitry architectures that can result in these characteristics in a compatible way.

However, these approaches focus on static stimuli and further assume the effect's isotropy and circular symmetry. 

Recent work in mice has revealed that feedback connections from MT to V1 do not target lower-order areas irrespectively of the tuning of the higher-order sending areas. This study portrays that feedback mainly targets lower-order neurons that respond to the same positions in the visual field as the former, but that a subset of these connections are in fact dispersed to other neurons: If a higher-order neuron is tuned to a particular direction, it presents a bias to connect to a lower-order neuron that responds to the visual field positions that would appear \textit{before} in that direction line. Similarly, if a high-order neuron is orientation tuned, it presents a bias for connecting to lower-order neurons that map visual field regions at orthogonal zones to that high-order neuron's preferred orientation. 

Given the putative relation between feedback and SM, this suggested possible anisotropies in the SM effect itself, possibly to relate and to put at the perspective of the viable circuitry involved.

Here, we present an extensive characterization of SM in transgenic mice's primary visual cortex performed with two photon imaging techniques that allowed access to large datasets of differently tuned neuronal responses. We developed a moving grating stimuli presentation protocol of multiple combinations of movement directions and spatial locations that enabled the study of the spatial structure of SM and its nonuniformity.

%The mapping of moving gratings properties and the spatial structure of surround modulation can produce a set of functional rules determined in the visual cortex. It shall further provide insight into feedback mechanisms and the perception of visual scenes. 

%Precluding the project, in here we review the main related theoretical, experimental and computational topics.

%The system comprises peripheral receptors as well as central neurons. From input neurons in the retina that receive the physical image information, signals follow the visual pathway connecting to the brain, where features from the current state of the environment are parallel processed and then hierarchically repeated at higher level stages. In here, based on previously learned information of working logic inferences, the mind makes new conjectures, educated guesses of high success probability about what it is that we could be "seeing", much analogously to an extraordinarily efficient and tuned machine learning algorithm. There's no one-to-one unequivocal complete copy of the world outside within our percepted reality - But, contrary to our intuition, perhaps sensitive sophisticated guesses are just as good - as well as biologically exequivable.
%The unified binding of the parallel results of our cognition processes is presumed to amount to a conscious perception of a given state of the sensorial world.

%To investigate this complex scheme, its stages and underlying principles must be described and interpreted. 
%The purpose of this project revolves around the surround modulation effect: the finding that, if a stimulus is present in the receptive field (RF), then its surround does influence the resulting signal form of action potentials. 

%We propose the study of the spatial structure of surround modulation with motion visual scenes. This means, to analyze the effects of the location of stimuli grading patches, with varied orientations and movement directions, around the RF in the ring-shaped surround of multiple neurons of different orientation \textit{selectivity} (see section 2.2).

%Following this objective, recordings of large populations of visual cortex neurons will be obtained using two-photon microscopy in transgenic mice expressing genetically encoded calcium indicators while they observe different sets of flanking stimuli. 

%Various multidisciplinary concepts are crucial to conduct this task, and these will be motivated and exposed in the present document.

%In the following section \ref{Theoretical}, visual neuroscience main notions are described, to introduce the relevant theoretical concepts. In section \ref{Plan}, the experimental previous methods on the matter of SM are also regarded, motivating this proposal's goals and novelty. The detailed experimental procedures this project aims will then be established as to answer remaining questions about SM. This document then accounts for the relevant experimental and analytical technologies, in section \ref{Experimental}: First, an explanation is provided on behaviour measurement control settings and a programmed stimuli display protocol, as well as the further adaptations it shall undergo for this project. Secondly, the relevant principles in chemical indicators and optical physics mechanisms to use for the expression and detection of neural activity will be clarified. Finally, the analysis of the experimental data will require appropriate methods for the treatment of a large number of individual cells - We will present a possible working approach to this, Suit2p. This report will then follow with a commented bibliography, section \ref{Bib}, the calendarization of the thesis work, section \ref{Cal}, and a final discussion of the general project's plan overview, section \ref{Con}.

%\\In the following state of the art section, as a theoretical introduction, we will start by presenting visual neuroscience main general notions (section 2.1), and then specify the current understanding of the retina's neurophysiology (section 2.2) and visual cortex (section 2.3). We will then enunciate the perception organization central ideas (section 2.4) and finally review the modern surround modulation investigation results, particularizing for V1 (section 2.5).

%%other principles in neurobiology are relevant, as well as computation familiarity with Matlab, $C^{++}$, specific behavior control and measurement packages and extensions, as well as optical physics and instrumentation engineering fundaments. These will be delineated
%\\The experimental phases of this project, will be delineated starting with a section for the stimuli and software development programming (section 3.1), a review on the sort of chemical indicators, genetic manipulation and physical mechanisms used for this experiment's expression of neural activity (section 3.2) and finalized by a presentation of the recording optical technology, two-photon excitation microscopy (section 3.3).

%\\Finally, the analysis of the experimental data will require appropriate methods for the treatment of great amounts of possible parameter configurations, each for a large number of individual cells - We will present a possible working approach to this, Suit2p (section 4).

%This report will follow with a commented bibliography (section 5), the calendarization of the to-be-done work (section 6) and a final discussion of the general project's plan overview (section 7).