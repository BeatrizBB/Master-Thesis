\section{Original Contributions}
\label{sec:int_contributions}

This project can be described in two sequenced components:

\begin{itemize}
\item Stimuli display software development with bpod system for three protocols: RF mapping, tuning assessment - both adapted from the previously used bcontrol system - and SM spatial structure study. Integration of the system within the laboratory's hardware.

\item Experimental sessions with the three stimuli presentation protocols, followed by a tuning protocol validation and analytical focus on RF mapping and SM spatial structure investigations.
\end{itemize}

Within the first frame, each created protocol is made available for multiple configurations of stimuli display. The user has the flexibility to change these stimuli parameters, covering a large range of stimuli types and experimental aims.

The experimental and analytical component of the project resulted in SM spatial structure findings, for mice V1 layers 2/3. Considering that each surround patch covered a quarter-annulus around the center stimulation and that gratings could move in any of four cardinal directions:

\begin{enumerate}
\item when center and surround stimuli are simultaneously shown, there is suppression of responses in most visually responsive cells, for any stimuli configuration.
\item this suppression is significantly larger ($10.8\%$) when two surround patches are displayed, instead of one (number effect).
%\item In any of the pooled conditions, going from one surround patch to two surround patches, both increased the surround 
\item there was no differential effect of SM with left versus right surround patches. The data was not conclusive for a top versus bottom surround patches effect (position effect).
\item there was a significant effect of increased suppression with shorter distances from the RF center to the surround stimuli, in the horizontal axis of the display, even whilst the cells did not respond to any of the surround-only conditions (distance effect).
\item No significant difference was found for SM strength with surround gratings going in up versus down nor in temporal versus nasal directions, with a trend for nasal suppressing more than temporal (direction effects).
\item Comparing surround gratings moving horizontally versus vertically held no significant difference, with a trend for horizontal gratings' orientation corresponding to stronger suppression than vertical orientation (orientation effect).
\item suppression is significantly stronger (with double surround, $5.3\%$ when surround stimuli is presented in positions collinear to its direction of movement than when presented in flanking positions regarding that direction (orientation alignment effect).
\item suppression is significantly stronger (with double surround, $4.3\%$) when presenting surround stimuli iso-directed to the center stimulus direction than when showing it cross-directed to the center stimulus direction (surround-center relative orientation effect).
\item the orientation of the center surround patch in regards to the cell's orientation selectivity held a significant influence over the distribution of the suppression indexes, so that, when center stimuli was with the anti-preferred orientation of the cell, this produced, for some conditions, higher suppression than when the center was with the preferred orientation. 
\item there were significant interaction effects between the collinear versus flanking effect and the iso-oriented versus cross-oriented effect.
\end{enumerate}

In addition to these, the obtained data portrayed other trends which did not prove statistically significant. Additional data is required to discard or validate those hypotheses, in particular for direction-related comparisons:

\begin{itemize}
\item surround stimuli moving towards the center stimuli versus outwards: data shows a barely significant effect for stronger suppression with the former.
\item surround stimuli and center stimuli moving in convergent versus divergent directions: data does not hold significance.
\item two surround patches moving in the same versus opposite directions to the center direction: data does not hold significance.
\end{itemize}